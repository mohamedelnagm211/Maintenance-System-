\documentclass[25pt]{article}
\usepackage{arabtex}
\usepackage[utf8]{inputenc}
\usepackage[LFE,LAE]{fontenc}
\usepackage[arabic]{babel}
\usepackage{fullpage}
\title{\Huge\centering{الصيانة}}
\author{عمر محمد}
\date{6102-21 مارس }



\begin{document}
	\maketitle
	\section{نظرة عامة}
	- هذا النظام مُصمم لشركات الصيانة ويهدف الي تسهيل اتمام عمليات التصليح بشكل بسيط بحيث يستطيع العميل التواصل مع الشركة و تتبع مراحل اصلاح جهازة بطريقة سلسة وبسيطة و من ناحية اخري يسهل علي الشركة تنظيم الطلبات و ادارة الوقت مع مواعيد العمل مما يوُدي الي تسريع عملية الرد علي العملاء وسهولة التواصل معهم و حل مشاكل العملاء في اسرع وقت ممكن .  
	\newline
	\large\section{مستخدمين النظام}
		\begin{enumerate}
			\item رئيس الشركة
			\item المدير 
			\item موظف الخدمات
			\item فني التصليحات 
			\item العميل
		\end{enumerate}
	\section{تفاصيل كل مستخدم}
	\begin{itemize}
		\item مدير النظام
		\begin{itemize}
			\item المعلومات المطلوبة
			\begin{enumerate}
				\item\textLR{UserName (ID@CompanyName.com)}
				\item\textLR{Password}
			\end{enumerate}
			\item الصلحيات
			\begin{enumerate}
				\item اضافة موظف الخدمات
				\item مسح موظف الخدمات 
				\item اضافة فني
				\item مسح فني 
				\item البحث عن موظف
				\item الاطلاع علي اداء الموظفين 
				\item الاطلاع علي الشكاوي 
				\item عمل حظر مواقت لاحد الموظفين
				\item متابعة حركة العمل في جميع الفروع
				\item  الاطلاع علي الاحصائيات مثل احصائيات التقدم وعدد الشكاوي ... الخ 
				\item الاطلاع علي شكاوي الموضفين و الرد عليهم
				\newline
			\end{enumerate}
			\item وصف عام 
			\newline
			هو المسئول عن النظام بشكل كامل فيستطيع توظيف الاشخاص و الاطلاع علي طلبات الموظفين وحل مشاكلهم وادارة الفروع 
		\newline
		\newline
		\end{itemize}
		
		\item موظف الخدمات
		\begin{itemize}
			\item المعلومات المطلوبة
			\begin{enumerate}
				\item\textLR{UserName (ID@CompanyName.com)}
				\item\textLR{Password}
				\item الرقم القومي
				\item السن
				\item الجنس
				\item تاريخ الميلاد
				\item العنوان 
				\item رقم التلفون
				\item رقم التعريف \textLR{ID}
				\item المرتب
				\item تاريخ التعيين 
				\item التخصص
			\end{enumerate}
			\item الصلحيات
			\begin{enumerate}
				\item الاطلاع علي طلابت  العملاء قيد الانتظار
				\item تحديد الوقت المناسب للعميل لاتمام عملية الصيانة
				\item تنظيم التواصل بين الفني و العميل
				\item حل الشكاوي 
				\item الرد علي العميل عند الستفسار علي شبئ
				\item تصعيد الشكوي لمدير النظام  
				\newline
			\end{enumerate}
			\item وصف عام 
			\newline
هو حلقة الوصل بين العميل و الفنيين و بين الفنيين و مدير النظام فهو يدير عملية الاصلاح بشكل اساسي فعند استلام طلب للصيانة فأنة يحدد مستوي التصليح ويحدد مع العميل موعد مناسب لة وتحديد فني للذهاب الي العميل في الوقت المتفق علية وتسجيل عمليات الاصلاح التي تمت او لم تتم وهو المسئول عن الاطلاع علي الشكاوي وحلها
			\newline
			\newline
		\end{itemize}
		\item الفني
		\begin{itemize}
			\item المعلومات المطلوبة
			\begin{enumerate}
				\item\textLR{UserName (ID@CompanyName.com)}
				\item\textLR{Password}
				\item الرقم القومي
				\item السن
				\item الجنس
				\item تاريخ الميلاد
				\item العنوان 
				\item رقم التلفون
				\item رقم التعريف \textLR{ID}
				\item المرتب
				\item تاريخ التعيين 
				\item التخصص
			\end{enumerate}
			\item الصلحيات
			\begin{enumerate}
				\item تنفيذ الاوامر القادمة من الخدمات
				\item تقديم الشكاوي  للخدمات   
				\item الاطلاع علي معلوماتة الشخصية 
				\item تسجيل الة الجهاز الذي تم تصليحة
				\newline
			\end{enumerate}
			\item وصف عام 
			\newline
			هو المسئول عن الذهاب للعميل واتمام عملية التصليح و التأكيد علي تصليح العطل 
			\newline
			\newline
		\end{itemize}
		
	\item العميل
	\begin{itemize}
		\item المعلومات المطلوبة
		\begin{enumerate}
			\item\textLR{UserName (Email@CompanyName.com)}
			\item\textLR{Password}
			\item بريد الكتروني
			\item العنوان 
			\item رقم التلفون
			\item رقم المنزل
		\end{enumerate}
		\item الصلحيات
		\begin{enumerate}
			\item تسجيل \textLR{Regist}
			\item تسجل الدخول
			\item التقديم علي طلب تصليح
			\item تتبع مدي تقدُم الجهارز 
			\item  تقديم شكوي للنظام
			\item تقيم مستوي الخدمة عقب كل عملية تصليح
			\newline
		\end{enumerate}
		\item وصف عام 
		\newline
		يتعامل مع النظام عن طريق التسجيل في الموقع وتقديم طلب للتصليهواختيار وقت مناسب لفني التصلحات  للقيام بعملية الصيانة وتقيم كل عملية اصلاح تمت من خلال حسابة	
		\newline
		\newline
	\end{itemize}
	\end{itemize}
	
	\section{كيفية التعامل مع النظام}
	\begin{itemize}
		\item العميل
		\newline
		في البداية يجب علي العميل التسجيل في الموقع حتي يستطيع طلب الخدمة  فيجب علية ملئ المعلومات المطلوبة { الاسم الاول , الاسم الثاني , العنوان , رقم المحمول , رقم المنزل , البريد الالكتروني , الرقم السري , سؤال الحماية} .
		\newline
		ثم عند اتمام عملية التسجيل يستطيع الان الدخول علي صفحتة الشخصية عن طريق كتابة "البريد الالكتروني , الرقم السري" ... وفي حالة نسيان الرقم السري سيتم سؤالة عن سوال الحماية ثم يستطيع كتابة رقم سري جديد .
		\newline 
		يسطتيع العميل الان طلب خدمة الصيانة عن طريق الضعط علي زر وملئ المعلومات المطلوبة " نوع الجهاز , الماركة , وشرح مبسط عن العطل , العنوان(اتركها فارغة اذا كان نفس العنوان) , التلفون (اتركها فارغة اذا كان نفس التلفون) " وفي حالة اكثر من جهاز فاضعط علي كلمة "اضافة جهاز" و املئ بعض المعلومات عن الجهاز الاخر,
		ثم الضعط علي زر ارسال (سيتم الرد عليك خلال 24 ساعة) .
		\newline 
		عن الرد سوف يرُسل للعميل اكثر من ميعاد حتي يسطتيع اختيار الانسب لة وعند اختيار الميعاد المناسب والضغط علي موافق سيتم الرد علي العميل خلال بضع ساعات وسوف يرُسل للعميل معلومات عن الفني الذي سيأتي الية مثل "الصورة , الاسم , الجنس , رقم هاتفة " وميعاد الحضور "التاريخ و الساعة" .
		\newline 
		في حالة نقل الجهاز االي مركز الصيانة يسطتيع العميل تتبع جهازة ومدي قدمة و ميعاد الاستلام .
		\newline
		\newline  
		وفي حالة وجود اي شكوي يمكن التواصل مع الشركة من خلال تقديم الشكوي فيجب كتابة الشكوي في المكان المحدد والضفط علي ارسال .
		\newline
		كل عميل يمتلك صندوق رسائل يسطتيع من خلالة استقبال الرسأل الخاصة بالرد علي الشكاوي و استلام اي معلومات جديدة خاصة بطلباتة او اختيار ميعاد ما او ابلاغ عن اي شء جديد خاص بجهازة .
		\newline 
		عند اتمام عملية التصليح يمكن للعميل ملي الفورم 
		الخاصة بجودة التصليح و مستوي الخدمة و راية الشخصي في الخدمة بشكل عام من حيث سرعة الاستجابة و كفاءة الفنيين و اعطاء تقيم من 1 - 10 الي الشركة وتقيم لل فني
		\newline
		\newline
		\item الفني
		\newline
	يجب علي الفني في البداية تسجيل الدخول عن طريق استخدام               \newlineبريد العمل( \textLR{ID@companyName.com}) و الرقم السري وفي حالة نسيان كلمة السر يجب الذهاب مدير النظام لتغيرة .
	\newline
	عند الدخول للصفحة الشخصية للفني فسبجد لدية في صندوق الوارد الطلبات التي يجب علية تنفيذها وعند استلامة الطلبات يحب الضغط علي زر استقبال وستتحول لدية من وضعية الاستقبال الي وضعية الانتظار حتي تكتمل .
	\newline
	عند الضعط علي استقبال سوف تختفي من صندوق الوارد وتظهر في صندوق النتظار فيستطبع الان الدخول الي صندوق الانتظار و رؤية تفاصيل كل طلب { العنوان , اسم العميل , رقم التلفون , رقم تلفون المنزل , العطل , نوع الجهاز , الماركة } .
	\newline
	عند اتمام عملبة التصليح فيجب الدخول علي صندوق الانتظار و الضغظ علي اكتمال و يجب ملئ بعض البيانات {التكلفة , قطع الغيار} .. وفي حالة عدم اكتمال التصليح يجب اختيار امر من الاثنين (الذهاب للمركز للتصليح , عدم توافر قطه الغيار) فسيتم اختفائة من الصندوق و ارسالة مرة اخري للخدمات .
	\newline
	ويمكن للفني ان يري تفاصيل حسابة من مستواة وتقديراتة ويمكن ان يري مَن افضل فني يعمل معة .
	\newline
	ويمكن تقديم شكاوي الخاصة بة عن طريق ارسال شكوتة الي مدير النظام عن طريق ملئ قورم الشكوي و الضفط علي ارسال .
	\newline
	\newline 
	
	\item موظف الخدمات
	\newline
		يجب علي موظف الخدمات البداية تسجيل الدخول عن طريق استخدام               \newlineبريد العمل( \textLR{ID@companyName.com}) و الرقم السري وفي حالة نسيان كلمة السر يجب الذهاب مدير النظام لتغيرة .
	\newline
	عند اكتمال عملية الدخول سوف يظهر لموظف الخدمات الطلبات قيد الانتظار وهناك اكثر من نوع من الطلبات 
	\begin{enumerate}
		\item\textLR{(waiting)} هذا النوع يعبر عن الطلبات التي لم يتم الاطلاع عليها من قبل موظف الخدمات .
		\item\textLR{(Seen notTecnical)} هذه معناها ان تم رؤايتها من قبل الموظف ولكن لم يحدد لها ميعاد ولا الفني الذي سيذهب .
		\item\textLR{(WaitCustomer)} وهذة معناها انها في انتظار رد العميل للموافقة علي الميعاد .
		\item\textLR{(checked)} وهذة معناها ان تم تحديد لها موعد و فني ولاكن لم تكتمل .
		\item\textLR{(uncompleated)} وهذه معناها ان العميل ذهب ولكن لم يكتمل التصليح .
		\item\textLR{(Done)} معناها ان الجهاز تم تصليحة .
		\newline
	\end{enumerate}
		 فعند دخول الموظف لرواية الطلبات ال \textLR{waiting} سوف يضغط علي احدي الطلبات و يري نوع الجهاز و العطل  .. الخ و يضغط علي زر رفض اذا كان النوع لا يتوافق مع امكانيات الشركة و قبول اذا كان العطل ممن الممكن تصليحة فعند الضغط موافق سوف يختفي من قائمة ال \textLR{waiting} الي قائمة \textLR{see nonTecnical} .
		 \newline
		 وبعد الاطلاع علي الطلبات سوف ينتقل الي مرحلة الاكتمال وهي ان يذهب الي صندوق ال \textLR{see nonTecnical} ويختار احدي الطلبات ثم يرسل لهم 3 مواعيد مختلفة في ايام مختلفة حتي يختار العميل ما يناسبة من هذة المواعيد .
		 \newline
		 عند الرد من قبل العميل سوف تظهر للموظف كي يحدد لها فني و سوف يتيح النظام رؤية مواعيد الفنيين و والوقت المتاح لكل منهم ويتحول الطلب الي \textLR{Checked} .
		 \newline
		 وعند اكتمال التصليح من قبل الفني سوف يتم نقلها الي حالة ال \textLR{Done} وان لم تكتمل سوف توضع في صندوق ال \textLR{uncompleated} ويمكنة من خلال هذا الصندوق ارسال فني اخر في وقت اخر او تحديد وقت انتهاء صيانتها .
		 \newline
		 ويوجد ايضا صندوق شكاوي يمكن للموظف الاطلاع علية لكي يري الشكاوي ويحاول حلها من خلال ارسال فني اخر .. الخ .
		 \newline
		 وفي حالة الاتصال من خلال التليفون سوف يتم تسجيل الطلب تحت مسمي \textLR{(PhoneOrder)} وسوف يتم تخزين معلومات الطلب حتي يتم التعامل معة من خلال النظام .
	\newline
	\newline
	\item مدير النظام
	\newline
	مدير النظام لة كلمة مرور و اسم مستخدم فريدين فيسطتيع الدخول للنظام و التحكم في كل ما يخص الموظفين .
	\newline
	فيسطتيع اضافة موضف من خلال ملئ الفورم الخاصة بة وتحديد صلحياتة من اختيارة ك فني او موظف خدمات و يمكن لة ايضا ان يبحث عن موضف و عن عميل  ويمكنة حظر حساب اي منهم .
	\newline
	ويستطيع ان يري مدي تقدم العمل و مدي تاخرة و الوامل المساعدة علي تاخير العمل و كمية الارباح و يستطيع الوصول لاي فرع والاطلاع علي احصائياتة وعن مدي كفاءة الموضفين .
	\newline 
	ويمكنة ايظا رواية تفاصيل اي موظف ورؤية مدي خبرتة وراي العملاء فية ويستطيع ان يري مقدار الطلبات التي تم تنفذها من خلالة ومدي سرعة استجابتة للطلبات .
	 \newline
	 ويمكنة ايضا ان يطلع علي شكاوي الموظفين وحلها و الموافقة علي طلباتهم و رفضها .
	 \newline
	 \newline
	\end{itemize}
	\section*{مهمات النظام}
	هذا النظام ينظم الطلبات بشكل تلقأي من حيث الاولوية وينظم للفنيين مواعيدهم و اماكن تواجدهم ويوفر للموظف المواعيد المتاحة للعميل فيرسلها للعميل اتلقائيا وعند رد العميل سوف يغير النظام كل الاشاء المعتمدة علي الهذا الميعاد  . 
	
\end{document}

